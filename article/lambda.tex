\documentclass[a4paper,adobefonts]{ctexart}
\usepackage{amsmath,amsthm,amssymb}
\usepackage[colorlinks=true,allcolors=black]{hyperref}
\usepackage{indentfirst}
\usepackage[a4paper,left=2.5cm,right=2.5cm,bottom=2.5cm,top=2.5cm]{geometry}
\usepackage{graphicx}
\usepackage{subcaption}
\usepackage{url}
\usepackage{fontspec}
\setmainfont{Palatino}
\setmonofont[Scale=MatchLowercase]{Monaco}
\pagestyle{plain}
\punctstyle{kaiming}
\usepackage{unicode-math}
\setmathfont{Asana Math}
\graphicspath{{pic/}}

\begin{document}
\title{{\bfseries 让我们谈谈 $\lambda$ 演算}}
\author{\href{mailto:txyyss@gmail.com}{王盛颐}}
\date{}
\maketitle

\section*{缘起}

``写篇介绍 $\lambda$ 演算的文章''这一条目在我的待办事项列表里已经躺了
很久了。我本科看``计算机程序的构造和解释''(SICP)一书时,接触到了
$\lambda$ 表达式这个概念。当时我觉得它就是一种表示匿名函数的记法。有了
这个记法,把函数作为值来传递、返回以及组合都方便了很多,别的也没多想。

SICP 里面用的 Scheme 语言让我见识到了世界上还有一类叫 ``函数式'' 的程
序设计语言。搜索相关资料,都说 $\lambda$ 演算是函数式语言的基础。好奇
心害死猫啊,我自然是要看看 $\lambda$ 演算是怎么一回事的。一番了解之后,
我被它简单的形式和丰富的内涵震惊了,真是漂亮啊。尤其当我看到 Church 数
的时候,那种``啊哈''的惊艳感真是永生难忘。

后来我学了越来越多的关于计算机,特别是程序设计语言的理论,发现
$\lambda$ 演算从可计算性理论到形式语义到类型理论,无处不在,自己目前所
知不过皮毛罢了。刚好我最近因为学习 Haskell 语言,实现了一个简单的无类
型纯 $\lambda$ 演算解释器。于是我决定写篇初步介绍无类型纯 $\lambda$ 演
算的文章,有个解释器的好处是能一边介绍一边通过解释器演示规约求值的过程,
让相对抽象的 $\lambda$ 演算更加直观和易于理解。

\section{一点历史}

首先让我八卦一下 $\lambda$ 演算的简史。

函数是数学中一个非常基本的概念,出现在几乎所有的数学分支里。为了研究函
数的一般性质,美国数学家 Alonzo Church 在 1936 年发明了 $\lambda$ 演算
这样一个形式系统,并以它为工具解决了 David Hilbert 在 1928 年提出的判
定性问题 (Hilbert's Entscheidungsproblem),这比图灵用他的图灵机解决同
一问题还早几个月。后来证明 $\lambda$ 演算和图灵机是等价的。现在谈到可
计算理论的时候,通常还是拿图灵机来作模型,不怎么提 $\lambda$ 演算。
$\lambda$ 演算被发明出来的时候,没有计算机,更没有程序设计语言。

但好东西到底是好东西,又过了差不多 30 年(谢天谢地那时候已经有计算机和
  程序语言了),英国计算机科学家 Peter Landin 发现可以通过把复杂的程序
语言转化成简单的 $\lambda$ 演算,来理解程序语言的行为。这个洞见加上众
所周知的 John McCarthy 的 Lisp 语言,让 $\lambda$ 演算广为传播。现在不
论是各种实际的程序设计语言还是理论上的研究工作,$\lambda$ 演算都是一个
绕不过去的基本工具。这让我觉得历史到底是公平的,没道理这么漂亮的理论比
不过图灵机。

$\lambda$ 演算之所以这么重要,用 Benjamin C. Pierce 的话说在于它具有某
种``二象性'':它既可以被看作一种简单的程序设计语言,用于描述计算过程,
也可以被看作一个数学对象,用于推导证明一些命题。在这篇介绍 $\lambda$
演算的文章里,我也打算从两个方面来讲,先讲它作为数学理论这方面的内容。

%% \section{作为一种数学理论的 $\lambda$ 演算}



\end{document}
