\documentclass[a4paper,adobefonts]{ctexart}
\usepackage{amsmath,amsthm,amssymb}
\usepackage[colorlinks=true,allcolors=black]{hyperref}
\usepackage{indentfirst}
\usepackage[a4paper,left=2.5cm,right=2.5cm,bottom=2.5cm,top=2.5cm]{geometry}
\usepackage{graphicx}
\usepackage{subcaption}
\usepackage{url}
\usepackage{fontspec}
\setmainfont{Palatino}
\setmonofont[Scale=MatchLowercase]{Monaco}
\pagestyle{plain}
\punctstyle{kaiming}
\usepackage{unicode-math}
\setmathfont{Asana Math}
\graphicspath{{pic/}}

\begin{document}
\title{{\bfseries 让我们谈谈 $\lambda$ 演算}}
\author{\href{mailto:txyyss@gmail.com}{王盛颐}}
\date{}
\maketitle

\section*{缘起}

``写篇介绍 $\lambda$ 演算的文章''这一条目在我的待办事项列表里已经躺了
很久了。我本科看``计算机程序的构造和解释''(SICP)一书时,接触到了
$\lambda$ 表达式这个概念。当时我觉得它就是一种表示匿名函数的记法。有了
这个记法,把函数作为值来传递、返回以及组合都方便了很多,别的也没多想。

SICP 里面用的 Scheme 语言让我见识到了世界上还有一类叫 ``函数式'' 的程
序设计语言。搜索相关资料,都说 $\lambda$ 演算是函数式语言的基础。好奇
心害死猫啊,我自然是要看看 $\lambda$ 演算是怎么一回事的。一番了解之后,
我觉得如此形式简单而内涵丰富的东西,就是美的体现啊。尤其当我看到
Church 数的时候,那种``啊哈''的惊艳感真是永生难忘。

后来我学了越来越多的关于计算机,特别是程序设计语言的理论,发现
$\lambda$ 演算从可计算性理论到形式语义到类型理论,无处不在,自己目前所
知不过皮毛罢了。刚好我最近因为学习 Haskell 语言,实现了一个简单的无类
型纯 $\lambda$ 演算解释器。于是我决定写篇初步介绍无类型纯 $\lambda$ 演
算的文章,有个解释器的好处是能一边介绍一边通过解释器演示规约求值的过程,
让相对抽象的 $\lambda$ 演算更加直观和易于理解。

\section{一点历史}

一切要从上个世纪初,也就是 1900 年左右开始说起了。那时候数学界的气象是
这样的:希尔伯特刚刚提出他的 23 个问题,其中第 2 个问题问的是公理系统
的相容性;1901 年提出的,动摇了集合论基础的罗素悖论还是个很流行的话题;
至于哥德尔不完备性定理呢,大家都还不知道。所以在这之后的几十年很多数学
家和逻辑学家都在致力于给整个数学建立一个一致的公理基础。比如罗素和怀特
海德写了``数学原理''(Principia Mathematica);比如约翰·冯·诺伊曼
\footnote{对,你没有看错,他就是那个定下了所有现代计算机基本结构``冯·
  诺伊曼结构''的那个冯·诺伊曼。}写了关于公理化集合论的博士论文;再比
如阿隆佐·邱奇(Alonzo Church)提出了 $\lambda$ 演算。

邱奇是个美国逻辑学家,生于 1903 年。他在 1928 年开始构造的一个形式系统
中包含了纯 $\lambda$ 演算。他发明这一形式系统的初衷是为了给逻辑学提供一
个基础,能代替罗素的类型理论和恩斯特·策梅洛(Ernst Zermelo)
\footnote{他就是 ZFC 公理化集合论的那个``Z''。}的集合理论。这个系统
1932 年发表后不久就被发现有矛盾,于是一年后邱奇修正了一番重新发表。他
希望那时发现不久的哥德尔关于``数学原理''一书不完备性定理不会扩展到他的
系统上。当时的人对不完备性定理威力有多大还缺乏清醒的广泛的认识。

愿望是良好的,结果是残酷的。到 1935 年,邱奇的两个学生,Stephen Kleene
和 Barkley Rosser 发现邱奇的逻辑系统是不一致的。但柳暗花明又一村,他们
发现系统包含的纯 $\lambda$ 演算则具有一系列良好的性质,再后来更是证明
用 $\lambda$ 演算可以等价的定义出可计算函数,邱奇觉得能有效计算的函数
等价于 $\lambda$ 可定义性,这就是著名的``邱奇--图灵论题''。更一进步的,
邱奇用 $\lambda$ 演算证明了一阶逻辑不存在递归判定过程,这是对希尔伯特
提出的判定性问题(Entscheidungsproblem)的第一个否定性答案,这比用图灵
证明停机问题不可判定还要早几个月。

不过二十世纪三十年代在 $\lambda$ 演算方面的成果差不多也就这些了,再接
下来的 20 年都没有太多 $\lambda$ 演算方面的研究和进展。直到六十年代,
那时有了计算机,有了程序设计语言,有了计算机科学家。在 1965 年,英国计
算机科学家 Peter Landin 发现可以通过把复杂的程序语言转化成简单的
$\lambda$ 演算,来理解程序语言的行为。这个洞见可以让我们把 $\lambda$
演算本身看成一种程序设计语言。而众所周知的 John McCarthy 的 Lisp 语言,
更是让 $\lambda$ 演算广为传播。现在不论是各种实际的程序设计语言还是理
论上的研究工作,$\lambda$ 演算都是一个绕不过去的基本工具了。

$\lambda$ 演算之所以这么重要,用 Benjamin C. Pierce 的话说在于它具有某
种``二象性'':它既可以被看作一种简单的程序设计语言,用于描述计算过程,
也可以被看作一个数学对象,用于推导证明一些命题。在这篇介绍无类型纯
$\lambda$ 演算的文章里,我也打算从两个方面来讲,先讲它作为数学理论这方
面的内容。

\section{作为一种数学理论}

\end{document}
